\documentclass[11pt]{article}

\usepackage{geometry}
\usepackage{amssymb}
\usepackage{amsmath}
\usepackage{graphicx}
\usepackage{hyperref}
\usepackage{times}
\usepackage{natbib}
\usepackage{setspace}
\usepackage{titling}
\usepackage{url}
\usepackage{xurl}
\usepackage{minted}
\usepackage{enumitem}

\onehalfspacing

\begin{document}

% Custom title with code font and reduced space between title and date
\begin{center}
\texttt{\Large magma-vaults}

\vspace{0.5em} % Adjusts the space between the title and the date
\texttt{\large \today}
\end{center}

\begin{abstract}
    \texttt{magma-vaults} are a fork of CharmFi vaults for EVM
    \cite{charmfi-doc} adapted to the Osmosis chain with minor enchancements.
    \texttt{magma-vaults} allow you to automatically manage 3 liquidity
    positions for Osmosis Supercharged pools \cite{osmosis-supercharged-doc}.
    One full range one, one concentrated, and a third one using
    out-of-proportion balances. Users can configure different vaults with
    different parameters, make them permisionless or permisioned, and decide
    the exact liquidity that will go into each position.
\end{abstract}

\section{Introduction}
UniswapV3 introduces the concept of ticks to allow for concentrated liqudity
positions over custom ranges. Thus, we distinguish 2 spaces we want to work in:
A tick space, where all ticks live, and a price space, where all prices live. As
prices are logarithmic, we construct a linear tick space by taking logarithms.
Its in this way that UniswapV3 price function $\rho(t) = 1.0001^t$ is constructed
(equation (6.1) \citep{uniswapv3}), with the nice property that given the current
tick $t$ of a liquidity pool, any concentrated position equidistant to $t$ will
be balanced. That its, for any $\varepsilon > 0$, positions in tick range
$[t-\varepsilon, t + \varepsilon]$ with reserves $(x, y)$ will have $\rho(t) =
y/x$.

One of the emergent features in UniswapV3 is that you can use concentrated
liquidity positions as limit orders, with the problem that, due to their
logarithmic nature, as prices grow, tick precision decreases. To give
an example, if the current price of bitcoin was 80000USDC, the difference
between the 2 closest ticks is of almost 8USDC. For this reaon, Osmosis introduces
geometric tick spacing with additive ranges \cite{osmosis-price-function}, that
allows us to define an order book on top of the AMM with proper tick precision.
Its easy to verify that its price function is:
\[
\rho(t) = 10^{\lfloor t/9e6\rfloor - 6} \left(
	t+10^6\left(1 - 9e6\left\lfloor \frac t{9e6}\right\rfloor
	\right)
\right)
\]

Naturally, mapping the tick space and price space in this way breaks our nice
equidistant property. For this reason, \texttt{magma-vaults} does operations over
the price space instead of the tick space, and then takes the price function
generalized inverse to decide the ticks for the balanced base position and the
limit one. It is then trivial to prove that, given the current tick $t$ and the
current price $p := \rho(t)$, any position in the price range $[p/k, pk]$ will
be balanced (we keep $k > 1$ for simplicity). We refer to those factors $k$ as
\texttt{PriceFactor}s in the code, and they will be, with the the
\texttt{Weight} type, our fundamental building block to configure custom
strategies.

\section{Architecture overview}
Magma vaults manage up to 3 concentrated liqudity positions. A full range
position, a base balanced position, and a limit position. Both, the full range
position and the base position will be balanced, but as prices move, the base
position could become unbalanced. Thus we expose a \texttt{rebalancing}
functionality, that will burn all positions and create them again centered
around the new price. Moreover, its easy to see that the total vault reserves
wont always be in proportion, so any out-of-proportion reserves will be used
for the limit position. Actually, vault total reserves almost never will be in
proportion, but its still theoretically possible a vault could get rebalanced
to have no limit position. Thus, the core business logic of the vault contract,
i.e, the rebalancing procedure, will:
\begin{enumerate}
    \item Burn all active liquidity positions to withdraw total reserves $(X, Y)$.
    \item Calculate all the balanced reserves $(x, y)$ in the vault. Its trivial
        to verify that:
        $$
        (x, y) = \begin{cases}
            (X, Xp) &,\; Y/p > X\\
            (Y/p, Y) &,\; \text{otherwise}
        \end{cases}
        $$
    \item Calculate the reserves $(x_0, y_0)$ to put into the full range position.
        For this, the vault admin (user) will decide a liquidity
        \texttt{Weight} $w\in(0, 1)$. For simplicity, we dont allow for extreme
        value of $w$, but the user still could make the difference between $w$
        and extremes negligible. The reserves will be calculated in such way
        that, if $L_0$ is the liquidity of the full range position, and $L_1$
        is the liquidity of the base one, $L_0/(L_0 + L_1) = w$ will be
        invariant. We will elaborate on this computation in following sections.
    \item Calculate trivially the reserves $(x_1, y_1)$ to put into the base
        range position as $(x_1, y_1) = (x - x_0, y - y_0)$. Then, calculate
        the concentrated tick range as $[\rho^{-1}(p/k_1), \rho^{-1}(pk_1)]$, for
        current price $p$ and \texttt{PriceFactor} $k_1$ decided by the user.
        The inverse $\rho^{-1}$ will be discussed in following sections.
    \item Calculate trivially the reserves $(x_2, y_2)$ to put into the limit
        position as $(x_2, y_2) = (X - x, Y - y)$. Naturally, $x_2 = 0\lor y_2 = 0$
        is invariant, and trivial from $(2)$. We then simply compute the tick range
        for the limit position as:
        $$
        [t_a, t_b] = \begin{cases}
            [\rho^{-1}(p/k_2), \rho^{-1}(p)] &,\; x_2 = 0\\
            [\rho^{-1}(p), \rho^{-1}(pk_2)] &,\; y_2 = 0
        \end{cases}
        $$
        Where $p$ is the current price, and $k_2$ a \texttt{PriceFactor} decided by the user.
\end{enumerate}
Thus, the only remaining question is when to rebalance. For this, any vault can use up to
3 rebalancing strategies:
\begin{itemize}
    \item \texttt{admin}: Only the vault admin can rebalance the vault whenever they
        feel like. Thus, its up to the admin to decide the rebalancing strategy off-chain.
    \item \texttt{delegate}: Any delegate address, decided by the admin, can rebalance
        the vault whenever they feel like. This allows for the system extensionality with,
        for example, oracle integration to rebalance automatically.
    \item \texttt{anyone}: Anyone can rebalance the vault as long as the price
        has moved outside the range $(p/k, pk)$ (where $p$ is the last
        snapshoted price during rebalance, and $k$ a \texttt{PriceFactor}
        decided by the admin) and as long as enough time has passed (threshold also
        decided by the admin). This allows for fully permisionless vaults if the
        vault admin decides to burn its ownership.
\end{itemize}

\section{Full range reserves computation}
Charmfi Alpha Vaults allocate liquidity to the full range position with
the formula $L = w\sqrt{XY}$ \cite{charmfi-liquidity-computation}. But this
introduces an imprecision: As the base range size is variable, its liquidity
will also be. Thus, \texttt{magma-vaults}, to make easier to reason about fees,
ensures the liquidity of the full range position remains constant through the
invariant:
\[
w = \frac{L_0}{L_0 + L_1}
\]
Thus, for example, if we set $w = 0.5$ we will be sure that, during any period
of time, both positions will earn the same amount of fees (assuming of course that
the base position stays in range during that time).

Clearly, $L_0 = x_0\sqrt p$. We then express $L_1$ in terms of UniswapV3
trading curve (equation (2.1) of the paper \cite{uniswapv3}) by taking
liqudity on the right, and $k$ as the base range \texttt{PriceFactor}:
\[
L_1 = x_1\frac{\sqrt{p}\sqrt{pk}}{\sqrt{pk}-\sqrt{p}}
    = x_1\frac{\sqrt{pk}}{\sqrt k - 1}
\]
Substituting then into our $w$ invariant:
\begin{equation*}
\begin{split}
w &= \frac{L_0}{L_0 + L_1}\\
  &= \frac{x_0\sqrt p}{x_0\sqrt p + \frac{x_1\sqrt{pk}}{\sqrt k - 1}}\\
  &= x_0\left({x_0 + \frac{x_1\sqrt{k}}{\sqrt k - 1}}\right)^{-1}\\
  &= x_0\left(
    \frac{x_0(\sqrt k - 1) + x_1\sqrt k}{\sqrt k - 1}
  \right)^{-1}\\
  &= \frac{x_0(\sqrt k -1)}{x_0(\sqrt k -1) + (x-x_0)\sqrt k}\\
  &= \frac{x_0\sqrt k - x_0}{x_0\sqrt k -x_0 + x\sqrt k - x_0\sqrt k}\\
  &= \frac{x_0\sqrt k - x_0}{x\sqrt k - x_0} =w
  \stackrel{(A)}\implies x_0 = \frac{\sqrt k wx}{\sqrt k + w - 1}
\end{split}
\end{equation*}
Where at step $(A)$ we simply solve for $x_0$. And once we know $x_0$,
we of course can simply get $y_0$ as $y_0 = x_0p$, which ensures the
reserves to be in proportion. We shall now prove that $x_0 \leq x \leq X$,
as the fact will be helpful later:

\begin{equation*}
\begin{split}
\frac{\sqrt k wx}{\sqrt k + w - 1}\leq x\iff\\
\sqrt k w\leq \sqrt k+w-1\iff\\
\sqrt k(w-1)+(1-w)\iff\\
(w-1)(\sqrt k-1)\leq 0
\end{split}
\end{equation*}
But $(w - 1) \geq 0$ y $(\sqrt k -1) \leq 0$ as $w\in[0, 1]$ and $k\in[1,
\infty)$. This concludes the proof.

\break
\subsection{Computation security proof}
We now prove that the way $x_0$ is computed in the \texttt{magma-vaults} codebase
is secure. Thus considered the implementation:
\begin{minted}[fontsize=\small]{rust}
pub fn calc_x0(k: &PriceFactor, w: &Weight, x: Decimal) -> Decimal {
    if w.is_zero() { return Decimal::zero() }
    do_me! {
        let sqrt_k = k.0.sqrt();

        let numerator = w.mul_dec(&sqrt_k);

        let numerator = Decimal256::from(numerator)
            .checked_mul(x.into())?;                         // (1)

        let denominator = sqrt_k
            .checked_sub(Decimal::one())?                    // (2)
            .checked_add(w.0)?;                              // (3)

        let x0 = numerator.checked_div(denominator.into())?; // (4)
        Decimal::try_from(x0)?                               // (5)
    }.unwrap()
}
\end{minted}
The \texttt{do\_me!} macro only creates a \texttt{anyhow::Error} closure
and runs it, so we only need to ensure the commented lines wont produce
any errors:
\begin{enumerate}[label=(\arabic*).]
    \item Wont overflow as $(2^{128} - 1)^2 < 2^{256} - 1$.
    \item Wont underflow as $k\in[1, \infty)$, thus $\sqrt k - 1 = 0$ in the
        worst case.
    \item Wont overflow as $w\in[0, 1]$, and we just subtracted 1 from $\sqrt
        k$ .
    \item Could only produce a division by zero if $\sqrt k -  1 + w = 0$.
        Assume $k= 1$, then $\sqrt k - 1 + w = 0 \iff w = 0$, in which case
        $x_0$ computation is trivial (see first line of the implementation).
        Note that also $k = 1 \land w = 0$ produces a vault with generally idle
        capital and only limit positions (which the first version does not
        support). On the other hand, the division wont overflow because $x_0
        \leq x$, as stated earlier.
    \item Finally, the downgrade to 128 bits wont fail because, again, $x_0
        \leq x$, but $x$ fits in 128 bits by definition.
\end{enumerate}


\section{Price function inverse}
The only missing piece is the generalized inverse $\rho^{-1}$ that will let
us map every price to its closest tick in Osmosis. We will prove that such
function is:
\[
\rho^{-1}(p) = 10^{6-\lfloor\log p\rfloor} \left(
	p+10^{\lfloor\log p\rfloor}\left(
		9\lfloor\log p\rfloor-1
	\right)
\right)
\]
Of course, it could be the case that there was 2 different prices with
the same tick, under the inverse. Dualy, there could be 2 ticks that map
to the same price. Thats why by generalized inverse we mean 2 things:
\begin{enumerate}
    \item Generalized inverse property: $\forall t\in T: \rho^{-1}(\rho(t)) =
        t$, where $T\subset\mathbb Z$ is the tick space.
    \item Closeness property: $\forall t_0, t_1\in T$ st $t_1 > t_0$, we have
        that $\forall p\in [\rho(t_0), \rho(t_1)]$ its the case that
        $\rho^{-1}(p)\in[t_0, t_1]$. That is, given any 2 ticks, if we map them
        to the price space and pick any price in that range, that price tick will
        also be in the tick range after pulling it back.
\end{enumerate}
But its easy to check that the weaker form of the closeness property when $t_1
= t_0 + 1$ is equivalent to the general form. And by dualizing the weaker form,
that is, taking ranges of the form $[t_1 - 1, t_1]$, we can trivially prove that
the generalized inverse property holds. Thus, we only need to prove the weaker
closeness property to prove that $\rho^{-1}$ is properly defined.

%But still work in progress; The derivation is not important, whats important
%is to prove its a generalized inverse of osmosis price function.
%\subsection{Computation security proof}
%Work in progress. I already got the proof in a obsidian note, just need
%to review and formalize it.

\break
\bibliographystyle{plain}
\bibliography{references}

\end{document}
